\documentclass[journal]{IEEEtran}

% *** GRAPHICS RELATED PACKAGES ***
%
\ifCLASSINFOpdf
  \usepackage{graphicx}
  \graphicspath{ {./images/} }
  \usepackage{amsmath}
  \usepackage{caption}
  \usepackage{subcaption}
  \usepackage{hyperref}
\else
\fi

% correct bad hyphenation here
\hyphenation{op-tical net-works semi-conduc-tor}


\begin{document}
%
% paper title
% Titles are generally capitalized except for words such as a, an, and, as,
% at, but, by, for, in, nor, of, on, or, the, to and up, which are usually
% not capitalized unless they are the first or last word of the title.
% Linebreaks \\ can be used within to get better formatting as desired.
% Do not put math or special symbols in the title.
\title{DeepProbLog Tasks}
%
%
% author names and IEEE memberships
% note positions of commas and nonbreaking spaces ( ~ ) LaTeX will not break
% a structure at a ~ so this keeps an author's name from being broken across
% two lines.
% use \thanks{} to gain access to the first footnote area
% a separate \thanks must be used for each paragraph as LaTeX2e's \thanks
% was not built to handle multiple paragraphs
%

\author{Davide~Lusuardi,~223821,~\IEEEmembership{davide.lusuardi@studenti.unitn.it}% <-this % stops a space
}

% The paper headers
% \markboth{BIO-INSPIRED ARTIFICIAL INTELLIGENCE, UNITN, JULY~2022}%
% {}

% make the title area
\maketitle

% As a general rule, do not put math, special symbols or citations
% in the abstract or keywords.
\begin{abstract}
  DeepProbLog is an extension of ProbLog that integrates Probabilistic Logic Programming with deep learning by means of neural predicates. The neural predicate represents probabilistic facts whose probabilities are parameterized by neural networks.

  DeepProbLog is a framework where general-purpose neural networks and expressive probabilistic-logical modeling and reasoning are integrated in a way that exploits the full expressiveness and strengths of both worlds and can be trained end-to-end based on examples.

  The aim of this report is to show how to solve AI tasks that require the integration of high-level reasoning and low-level perception. We focus on the multi-digit MNIST octal-division task that consists in the division between two lists of MNIST digits representing multi-digit octal numbers. Using DeepProbLog we are able to solve the task given that supervision is only present at the output side of the probabilistic reasoner and considering that the approach can be extended to multi-digit numbers without being explicitly trained on them.  
\end{abstract}

% Note that keywords are not normally used for peerreview papers.
% \begin{IEEEkeywords}
% NEAT, Neural Networks, Neuroevolution, GP, Genetic Algorithm, Artificial Intelligence, Computer Games, Autonomous agent.
% \end{IEEEkeywords}


% For peer review papers, you can put extra information on the cover
% page as needed:
% \ifCLASSOPTIONpeerreview
% \begin{center} \bfseries EDICS Category: 3-BBND \end{center}
% \fi
%
% For peerreview papers, this IEEEtran command inserts a page break and
% creates the second title. It will be ignored for other modes.
\IEEEpeerreviewmaketitle



\section{Introduction}
% \IEEEPARstart{T}{here} are many tasks in AI that require both low-level perception and high-level reasoning but the integration of the two is an open challenge in the field of artificial intelligence.
\IEEEPARstart{A}{rtificial} intelligence tasks often require both low-level perception and high-level reasoning, but integrating the two remains an open challenge in the field.
Today, low-level perception is typically achieved by deep neural networks, while high-level reasoning is typically handled using logical and probabilistic representations and inference. Even if deep learning can create intelligent systems used to interpret images, text and speech with unprecedented accuracy, there is a growing awareness of its limitations: deep learning requires large amounts of data to train a network and the models are black-boxes that do not provide explanations and cannot be modified by domain experts. 

The abilities of deep learning and probabilistic logic approaches are complementary: deep learning excels at low-level perception and probabilistic logic excels at high-level reasoning. Recently, there has been a lot of progress in both deep learning and high-level reasoning areas and today there exists approaches able to integrate logical and probabilistic reasoning with statistical learning.
% TODO: in order to exploit strenghts and weakenesses of both

DeepProbLog \cite{DeepProbLog} is one possible approach. It is a neural probabilistic logic programming language that incorporates deep learning by means of neural predicates. With DeepProbLog, instead of integrating reasoning capabilities into a complex neural network architecture, the authors have decided to start from an existing probabilistic logic programming language, ProbLog \cite{ProbLog}, that has been extended with the neural predicates. In this way, the framework exploits the full expressiveness and strengths of general-purpose neural networks and expressive probabilistic-logical modeling and reasoning and can be trained end-to-end based on examples.

In this report, we focus on the application of DeepProbLog to the multi-digit MNIST octal-division task, a computer vision problem that involves dividing two multi-digit octal numbers. This task is challenging due to the large number of possible input-output pairs and the need to apply high-level reasoning to perform the division. Moreover the training set contains only single-digit numbers and the program is not explicitly trained on multi-digit numbers.

We begin by providing an overview of DeepProbLog and its key features in Section \ref{sec:DeepProbLog}. We then describe the multi-digit MNIST octal-division task in detail and explain how it can be formulated as a DeepProbLog program in Section \ref{sec:task}. Finally, in Section \ref{sec:results} we present experimental results that demonstrate the effectiveness of the DeepProbLog approach on this task.

% In this work, we introduce DeepProbLog explaining the basics of the language in Section \ref{sec:DeepProbLog}. Subsequently, in Section \ref{sec:task} we present the multi-digit MNIST octal-division task and how can be solved using DeepProbLog. Finally, in Section \ref{sec:results} we discuss the results obtained.

% TODO: finish introduction
Overall, our goal is to highlight the potential of DeepProbLog as a powerful tool for solving complex problems in artificial intelligence and machine learning, particularly those that require the integration of logical and probabilistic reasoning with statistical learning.
\input{02_task}
\input{03_methodologies_implementation}
\section{Results}
\label{sec:results}
% TODO: to conclude

\begin{figure}[t]
    \centering
    \begin{subfigure}[b]{0.45\textwidth}
        \centerline{\includegraphics[scale=0.4]{validation_accuracy.png}}
        \caption{Validation accuracy}
        \label{fig:acc}
    \end{subfigure}
    \hfill
    \begin{subfigure}[b]{0.45\textwidth}
        \centerline{\includegraphics[scale=0.4]{training_loss.png}}
        \caption{Training loss}
        \label{fig:loss}
    \end{subfigure}
    \caption{Multi-digit MNIST octal-division task learning curves: accuracy on the multi-digit test set \ref{fig:acc} and training loss on the single-digit training set \ref{fig:loss}.}
    \label{fig:training_curves}
\end{figure}

% In this section we briefly present the results obtained solving the multi-digit MNIST octal-division task.
In this section, we briefly present the results obtained evaluating the performance of the DeepProbLog framework on the multi-digit MNIST octal-division task.

We trained the model on a training set made of 22,958 pairs and evaluated it on a test set made of 1,462 pairs.
% Our experiments show that DeepProbLog achieves high performance on the multi-digit MNIST octal-division task.
The learning process managed to achieve $97\%$ of accuracy on the test set after around 15,000 iterations as shown in Fig. \ref{fig:acc}. It is worth noting that the process is able to reach a validation accuracy of about $90\%$ after just 5,000 iterations.
% The losses on the training
Fig. \ref{fig:loss} shows the loss obtained on the training set during the learning process: the loss significantly decreases during the first 5,000 iterations and then it tends to oscillate under the value of $0.2$ in agreement with the validation accuracy.
% The learning curve is shown in Fig. \ref{fig:loss}.

This suggests that DeepProbLog is able to effectively leverage the logical rules and the neural network, handling multi-digit numbers in the test set without being explicitly trained on them, converging faster and requiring a smaller training set compared to traditional neural network models. These results demonstrates the effectiveness of DeepProbLog in solving challenging Deep Learning problems.

\section{Conclusions}
\label{sec:conclusions}

As illustrated in the previous sections, DeepProbLog is a powerful framework where general-purpose neural
networks and expressive probabilistic-logical modeling and reasoning are integrated in a way that exploits the full expressiveness and strengths of both worlds. For these reasons, the framework is suitable for solving problems where both low-level data processing using deep networks and high-level reasoning using symbolic approaches are needed.

After an explanation of DeepProbLog, we have illustrated how to create and train a DeepProbLog program to solve the multi-digit MNIST octal-division task. Our experiments showed that DeepProbLog was able to accurately recognize the MNIST digits and perform the division operation accurately, even when presented with multi-digit numbers.
% Overall, we demonstrated the effectiveness of DeepProbLog in solving machine learning tasks that require both low-level perception and high-level reasoning.

Compared to standard neural classifiers, DeepProbLog requires fewer iterations to converge, due to the fact that exploits symbolic reasoning, and provides a transparent and interpretable framework, which is particularly important in applications where decisions have significant consequences.

% Moreover, we found that DeepProbLog required fewer iterations to converge compared to standard neural classifiers, which is likely due to the fact that DeepProbLog combines the efficiency of deep learning with the expressiveness of probabilistic programming.

% The results of our experiments demonstrate the effectiveness of DeepProbLog in solving AI tasks that require both low-level perception and high-level reasoning.
% Similarly, many more AI tasks that require both low-level perception and high-level reasoning can be efficiently solved using the DeepProbLog language.
% DeepProbLog neural probabilistic logic programming language.

% TODO: standard neural classifier requires more iterations to converge, multi-digit numbers in the test set, larger test set.

% Overall, our results demonstrate that DeepProbLog is a powerful framework for solving complex probabilistic modeling tasks and has the potential to advance the state-of-the-art in various domains, including computer vision, natural language processing, and robotics.

% Overall, our results highlight the potential of DeepProbLog in integrating probabilistic reasoning with deep learning techniques and in advancing the state-of-the-art in various domains, including computer vision, natural language processing, and robotics. Future work could explore the application of DeepProbLog to other complex modeling tasks and the development of more efficient optimization techniques.

Overall, DeepProbLog is a powerful framework for solving complex probabilistic modeling tasks, and it has the potential to advance the state-of-the-art in various domains, including computer vision, natural language processing, and robotics.


% Can use something like this to put references on a page
% by themselves when using endfloat and the captionsoff option.
\ifCLASSOPTIONcaptionsoff
  \newpage
\fi

% references section

% can use a bibliography generated by BibTeX as a .bbl file
% BibTeX documentation can be easily obtained at:
% http://mirror.ctan.org/biblio/bibtex/contrib/doc/
% The IEEEtran BibTeX style support page is at:
% http://www.michaelshell.org/tex/ieeetran/bibtex/
%\bibliographystyle{IEEEtran}
% argument is your BibTeX string definitions and bibliography database(s)
%\bibliography{IEEEabrv,../bib/paper}
%
% <OR> manually copy in the resultant .bbl file
% set second argument of \begin to the number of references
% (used to reserve space for the reference number labels box)
\begin{thebibliography}{1}

\bibitem{Yellow-Spaceship}
https://github.com/ph3nix-cpu/Yellow-Spaceship

\bibitem{NEAT}
K. O. Stanley and R. Miikkulainen, "Evolving Neural Networks through Augmenting Topologies," in Evolutionary Computation, vol. 10, no. 2, pp. 99-127, June 2002, doi: 10.1162/106365602320169811.

\bibitem{GP}
Koza, John R. "Hierarchical genetic algorithms operating on populations of computer programs." IJCAI. Vol. 89. 1989.

\bibitem{NEAT-Python}
https://neat-python.readthedocs.io/en/latest/index.html

\bibitem{DEAP}
https://deap.readthedocs.io/en/master/index.html

\bibitem{PyGame}
https://www.pygame.org/docs/

\bibitem{repository}
https://github.com/samuelbortolin/Bio-Inspired-Spaceship

\end{thebibliography}

\end{document}


